\chapter{Přílohy}

%sekce pro kod programu ? - Elektronicka priloha

\section{Dokumentace}

\subsubsection{Uživatelská dokumentace}

Uživatelská dokumentace je součástí webové stránky. Konkrétně je popsaná na úvodní stránce. Pro jednodušší přístup přikládáme uživatelskou dokumentaci do elektronické přílohy této práce.

\subsubsection{Programátorská dokumentace}

Programátorská dokumentace je obecně popsaná v~kapitole~\ref{kapitola-Implementace}. Dokumentace kódu je vytvořena pomocí XML komentářů. Samotný kód přikládáme v~elektronické příloze.


\subsection{Instalace}

\subsubsection{Prerekvizity}

Ke spuštění či překladu aplikace potřebujeme platformu .NET~5.0, která je k~dispozici na stránkách Microsoftu\footnote{Ke stažení na stránce \url{https://dotnet.microsoft.com/en-us/download/dotnet/5.0}}.

Instalace platformy .NET~5.0 nám zpřístupní příkaz \textbf{dotnet}, pomocí kterého aplikaci spustíme. Alternativně můžeme ke spuštění využít nástroje poskytované prostředím \textbf{Visual Studia}.

\subsubsection{Spuštění}

Aplikaci přeložíme zavoláním příkazu \textbf{dotnet build} z~adresáře obsahujícího aplikaci.

Příkaz \textbf{dotnet run -{}-project src/Web/Web.csproj} použijeme ke spuštění naší webové aplikace.

Pro ověření funkčnosti aplikace můžeme spustit \textbf{Unit Testy} pomocí příkazu \textbf{dotnet test}.


\subsection{Konfigurační soubor}

% popsat ze je urcen pro spravce webove aplikace, ktery ho meni

\subsubsection{GTFS data}

Konstanta \textbf{ShouldUpdate} určuje, zdali chceme při spuštění aplikace data aktualizovat, viz popis třídy \textbf{DataUpdater} v~sekci~\ref{trida-DataUpdater}. Při aktualizaci dojde ke stažení dat a~k~jejich deserializaci. Pokud data neaktualizujeme, serializujeme dříve uložená data.

URL pro stažení dat specifikujeme konstantou \textbf{GTFSSourceURI}. Aktuálně podporujeme jen data archivovaná ve formátu zip.

% mohla by byt podsekce `zpracovani dat`
Data stahujeme do složky určené konstantou \textbf{PathToGTFSFolder} a~cestu k~serializaci určuje konstanta \textbf{GTFSSerializationPath}.

Platnost dat závisí na konkrétním zprostředkovateli a~dá se nastavit skrze konstantu \textbf{ValidityInDays}.

UTC offset pro časy jízdních řádů, které uchováváme na serveru, lze nastavit konstantou \textbf{UTCOffset}.

\subsubsection{Přestupy}

Generování přestupů popisujeme v~sekci~\ref{generovani-prestupu}.

Konstantu \textbf{MaxTransferDistanceInMeters} používáme pro omezení vzdálenosti během generování přestupů.

Konstantu \textbf{WalkingSpeedInMetersPerSec} určuje průměrnou rychlost chůze. Používáme ji k~odhadu času potřebného pro přestup.

\subsubsection{Sousední zastávky}

Sousední zastávky detailně popisujeme v~sekci~\ref{class-NearestStops}.

Konstantu \textbf{WalkingSpeedInMetersPerSec} používáme také pro odhad času potřebného k~příchodu z~nějakého místa na zastávky v~okolí.

Konstanta \textbf{NearestStopsDistanceInMeters} omezuje vzdálenost, ve které hledáme z~daného místa sousední zastávky.

\subsubsection{Vizualizace bodů}

Konstanta \textbf{VisualisedRasterPointsResolution} určuje rozlišení, pro které vizualizujeme dostupnost v~bodech, viz sekce~\ref{optim-body}.