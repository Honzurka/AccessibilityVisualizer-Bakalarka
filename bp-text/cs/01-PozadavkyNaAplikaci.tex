\chapter{Cíle práce}

V~této kapitole si detailně popíšeme cíle, kterých se budeme snažit dosáhnout. Zbylé kapitoly budou řešit některé z~cílů uvedených zde.

\section{Terminologie}\label{data-terminologie}

\begin{itemize}
    \item \textbf{Dostupnost} definujeme jako dobu, za jakou se ze vstupního místa dostaneme na cílové místo. S~takto definovanou dostupností se nám, v~porovnání s~konkrétními časy, bude snáze pracovat ve statistických výpočtech.\label{definice-dostupnosti}
    
    \item \textbf{Místo} je nějaká pozice v~prostoru, která není příliš vzdálená od některé ze zastávek. V~této práci zásadně odlišujeme význam místa a~zastávky.

    \item \textbf{Zastávka} je místo, na kterém staví nějaký spoj.

    \item \textbf{Trip} překládáme jako \textbf{jízdu}. Jízda je posloupnost zastávek.
    
    \item \textbf{Route} překládáme jako \textbf{trasu}. Trasa je množina jízd jedoucích přes stejné zastávky.
    
    V~kapitole~\ref{kapitola-1} však pracujeme s~\textbf{GTFS trasou}, jejíž definice se liší od definice trasy, kterou používáme v~následujících kapitolách. GTFS trasy totiž mohou obsahovat jízdy, které jedou přes různé množiny zastávek.
    
    \item \textbf{Transfers} překládáme jako \textbf{přestupy}. Přestupovat můžeme mezi dvěma zastávkami a~přestup trvá nějaký čas.
    
    \item \textbf{Spoj} používáme pro nějaký dopravní prostředek, který jede v~konkrétní čas po zastávkách jízdy. %podobnost s TripWithDate v impl.
    
    \item \textbf{Journey} překládáme jako cestu. Cesta je posloupnost spojů a~přestupů, kterými se dopravíme z~počátečního místa na cílové.

\end{itemize}


\section{Požadavky na aplikaci}
% trosku to souvisi s implementaci ze sekce 5.4 kde aplikaci popisuju - i kdyz tam ale popisuju existujici design

\subsubsection{Frontend by měl:}
\begin{itemize}
    \item Poskytovat rozhraní pro zadávání vstupu uživatelem.
    
    Uživatel může zadávat více často navštěvovaných míst.
    
    Zadané místo musí být v~dostatečné blízkosti nějaké zastávky.
    
    S~místem lze asociovat důležitost, která určuje, jak moc je dostupnost z~tohoto místa zohledněna ve vypočtených dostupnostech.
    
    Uživatel může s~jedním místem asociovat více časů, ve kterých chce z~místa odjíždět.
    
    Uživatel má mít možnost nezadávat čas přesně, ale nechat aplikaci vyhodnotit dostupnost pro nějaké okolí zadaného času.
    
    \item Poskytovat rozhraní pro zobrazování výstupu uživateli.
    
    Na výstupu se uživateli vizualizuje vypočtená dostupnost.
    
    Dostupnost má být vizualizovaná na všech místech.
    
    Uživatel má být ve vizualizaci schopen snadno najít dostupnost pro konkrétní místa.
    
    \item Posílat požadavky na backend, který běží na serveru.
\end{itemize}

\subsubsection{Backend by měl:}
\begin{itemize}
    \item Zpracovávat dotazy z~frontendu v~rozumném čase. % maximálně jednotky sekund
    
    \item Aktualizovat data jízdních řádů.
    
    \item Delegovat vyhodnocení dostupnosti na knihovnu.
\end{itemize}

\subsubsection{Knihovna by měla:} %sjednoceni s backendem ???
\begin{itemize}
    \item Být nezávislá na webové aplikaci.
    
    \item Dostatečně obecná, aby mohla pracovat s~jízdními řády po celém světě.
    
    \item Načítat data jízdních řádů a~přetransformovat je do formátu vhodného pro vyhledávání.
    
    \item Pro daný vstup vyhodnocovat dostupnost na všech místech. % klicova cast prace
\end{itemize}


\section{Problémy k~řešení}


\subsubsection{Data}\label{problemyKReseni-data}

Nejprve potřebujeme mít přístup k~datům jízdních řádů. Bez samotných dat se nemůžeme pokoušet vyhodnocovat dostupnost.


\subsubsection{Vyhodnocení dostupnosti}\label{problemyKReseni-dostupnost}

Pro vyhodnocení dostupnosti musíme nejprve nalézt cesty, které vedou mezi vstupními a~výstupními místy. Tento problém si rozdělíme na následující podproblémy, které budeme postupně řešit.

\begin{itemize}
    \item Vyhledání cesty mezi 2 zastávkami. To je klasický problém, který řeší běžné vyhledávače.
    
    \item Vyhledání cesty mezi 1 vstupní zastávkou a~všemi ostatními zastávkami. Toto je důležité, neboť cílíme na vyhodnocení dostupnosti na všech místech. Vyhodnocení dostupnosti na všech zastávkách je tedy přirozený mezikrok.
    
    \item Zadávání více vstupních zastávek. Vstupní zastávky jsou ty, které uživatel zadává jako často navštěvovaná místa. Takových zastávek je, v~porovnání se všemi zastávkami, velmi málo. Tento mezikrok potřebujeme, abychom uživatelům umožnili zadávat více často navštěvovaných míst.
    
    \item Zobecnění cílové zastávky na místo. V~této fázi jsme již schopni vyhodnocovat dostupnost na všech zastávkách. Abychom byli schopni vyhodnocovat dostupnost na libovolných místech, potřebujeme zobecnit zastávky na místa.
    
    \item Zobecnění vstupní zastávky na místo. Toto zobecnění potřebujeme, neboť chceme našim uživatelům umožnit zadávat na vstupu libovolná místa, ne jen zastávky.
    
    \item Vyhledávání oběma směry. Předchozí podproblémy počítají se situací, kdy uživatelem zadaná místa jsou ta, ze kterých chce vyjíždět. Uživatele by však mohla zajímat situace, kdy na zadaná místa chce v~daný čas přijíždět. Tuto možnost mají některé vyhledávače, které umožňují vyhledávat spoje dle času dojezdu.
    
    \item Vyhodnocování dostupnosti pro časový interval. Vyřešení tohoto podproblému by umožnilo uživatelům nezadávat vstupní čas přesně. Mohli bychom totiž vyhodnocovat dostupnost pro nějaké okolí zadaného času.
\end{itemize}


\subsubsection{Vizualizace}\label{problemyKReseni-vizualizace}

Potřebujeme zajistit způsob vizualizace, který uživatelům umožní dohledat ideální místo podle dostupnosti.

Potřebujeme najít způsob, jak vizualizovat dostupnost pro všechna místa.

% usnadnili orientaci ve výsledcích a~umožnili jim snadné vyhledávání nejdostupnějších míst



% chybi prestupy? ---- to vsak zalezi i na datech, v tuhle chvili to tedy nejspise neni problem

% v dalsich kapitolach se chci odkazovat na pozadavky, ktere prave resim
