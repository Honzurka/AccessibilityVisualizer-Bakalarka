\chapter*{Úvod}
\addcontentsline{toc}{chapter}{Úvod}

% popis kontextu
Každý by chtěl bydlet v~blízkosti míst, která často navštěvuje. V~dnešní době se však většina těchto míst nachází ve větších městech kde jsou ceny bydlení vyšší. Ne každý má dostatek finančních prostředků k~tomu, aby mohl bydlet v~blízkosti navštěvovaných míst. Mnozí, zejména ti kteří cestují hromadou dopravou, pak tráví dennodenně dlouhé hodiny na cestách.

Lidé navíc často navštěvují různorodá místa, jako jsou třeba zaměstnání, škola, známí, lékaři či různé obchody. Ani pro člověka s~dostatkem finančních prostředků tedy nemusí být snadné naleznout ideální lokalitu, odkud by byl schopen dostat se na často navštěvovaná místa v~minimálním čase.

% podobne prace
V~současné době existuje mnoho způsobů, jak nalézt optimální cestu hromadnou dopravou mezi dvěma místy. Pokud vím, není k~dispozici vyhledávač, který by tento koncept rozšiřoval a~byl schopen nalézt optimální cesty mezi několika zadanými a~všemi ostatními místy.

% cile prace
Na základě tohoto rozšířeného konceptu se v~této práci pokusíme ohodnotit všechna místa podle dostupnosti veřejnou dopravou ze zadaných míst v~daných časech. Vypočtenou dostupnost vizualizujeme ve všech místech a~tím umožníme snadnou orientaci ve výsledcích. Pro zpřístupnění aplikace běžným uživatelům vytvoříme webovou aplikaci. Výsledky aplikace budou pro uživatele dopočteny v~přijatelném čase. Samotná aplikace bude dostatečně obecná, aby byla schopna pracovat s~jízdními řády z~různých lokalit.

% popis jednotlivych kapitol
V~první kapitole detailně popisujeme cíle práce.
Ve~druhé kapitole popisujeme data jízdních řádů a~jejich poskytovatele.
Ve třetí kapitole popisujeme námi zvolený algoritmus pro hledání cest v~jízdních řádech.
Ve čtvrté kapitole popisujeme vlastní úpravy algoritmu zvoleného v~předchozí kapitole.
V~páté kapitole zobecňujeme algoritmus z~předchozí kapitoly a~umožňujeme tak vyhledávat dostupnost z~několika vstupních míst na všechna ostatní výstupní místa.
V šesté kapitole popisujeme přístupy k~vizualizaci dostupnosti.
V~sedmé kapitole popisujeme implementaci myšlenek, popsaných v~předchozích kapitolách a~popisujeme také implementaci webové aplikace.
V~osmé kapitole si ukážeme, jak naše aplikace funguje s~jízdními řády z~různých lokalitách.
V~příloze dokumentujeme, jak se má naše aplikace zprovoznit a~jak ji lze nakonfigurovat.

% nepopisuji zaver - nevim jak moc je takovy popis potreba 